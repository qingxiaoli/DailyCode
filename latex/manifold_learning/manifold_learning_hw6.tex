\documentclass[a4paper, UTF8]{ctexart}
\usepackage{ctex}
\usepackage{amsmath}
\usepackage{amsthm}
\usepackage{multirow}
\usepackage{amssymb}
\usepackage{graphicx}
\usepackage{geometry}
\usepackage{bm}
\usepackage{subfigure}
\usepackage{float}
\usepackage{mathrsfs}
\renewcommand\thesection{\arabic{section}}
\newtheorem*{exercise}{\textbf{习题}}
\newtheorem*{theorem}{Theorem}
\title{Manifole Learning Homework 6}
\date{\today}
\author{安捷 1601210097}
\begin{document}
\maketitle
  \begin{exercise}[95]
    \begin{proof}
      由于 $k_1$ 是一个核函数,因此由Mercer定理,对于任意的 $L_2$可积函数 $g$有
      \begin{equation}
        \int g \left( \mathbf{x} \right) g \left(  \mathbf{y}\right)k_1 \left( \mathbf{x}, \mathbf{y} \right) d \mathbf{x} d \mathbf{y} > 0
      \end{equation}
      而由于 $k \left( \mathbf{x}, \mathbf{y} \right) = \mathrm{exp}\left( k_1 \left( \mathbf{x}, \mathbf{y} \right) \right) > k_1 \left( \mathbf{x}, \mathbf{y} \right)$,因此有
      \begin{equation}
        \int g \left( \mathbf{x} \right) g \left( \mathbf{y} \right) k \left( \mathbf{x}, \mathbf{y} \right) d \mathbf{x} d \mathbf{y} > \int g \left( \mathbf{x} \right) g \left( \mathbf{y} \right) k_1 \left( \mathbf{x}, \mathbf{y} \right) d \mathbf{x} d \mathbf{y} > 0
      \end{equation}
      由Mercer定理, $k \left( \mathbf{x}, \mathbf{y} \right)$ 是核函数。
    \end{proof}
  \end{exercise}
\end{document}
