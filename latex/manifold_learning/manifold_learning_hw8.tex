%!TEX program = xelatex
\documentclass[a4paper, UTF8]{ctexart}
\usepackage{ctex}
\usepackage{amsmath}
\usepackage{amsthm}
\usepackage{multirow}
\usepackage{amssymb}
\usepackage{graphicx}
\usepackage{geometry}
\usepackage{bm}
\usepackage{subfigure}
\usepackage{float}
\usepackage{mathrsfs}
\renewcommand\thesection{\arabic{section}}
\newtheorem*{exercise}{\textbf{习题}}
\newtheorem*{theorem}{Theorem}
\title{Manifole Learning Homework 7}
\date{\today}
\author{安捷 1601210097}
\begin{document}
\maketitle
	\begin{exercise}[145.1a]
		为了证明$f_1$是凸函数, 我们只需证明$f_1$满足如下凸集定义 
		\begin{equation} \label{cvx_def}
		f_1(\alpha \mathbf{x}_1 + (1 - \alpha) \mathbf{x}_2) \leq \alpha f_1(\mathbf{x}_1) + (1 - \alpha) f_1(\mathbf{x}_2),
		\end{equation}
		其中 $\alpha \in [0, 1] $. 显然,因为 $\alpha \geq 0$ 并且 $(1 - \alpha) \geq 0$, 如果存在元素$\mathbf{x}_1$ or $\mathbf{x}_2$ 小于 $0$, 那么如果 $\alpha \mathbf{x}_1 + (1 - \alpha) \mathbf{x}_2 > 0$, 定义 \ref{cvx_def} 满足; 如果 $\alpha \mathbf{x}_1 + (1 - \alpha) \mathbf{x}_2 \leq 0$, 那么 $+\infty \leq +\infty$ 也满足定义 \ref{cvx_def}.

		上面讨论了变量小于0的情况, $f_1$ 满足凸函数的定义. 如果 $\mathbf{x}_1 > \mathbf{0}$ 并且 $\mathbf{x}_2 > \mathbf{0}$, 我们有如下推导
		\begin{equation}
		\notag
		\begin{split}
		f_1(\alpha \mathbf{x}_1 + (1 - \alpha) \mathbf{x}_2) & = -\left[\prod_i (\alpha x_{1i} + (1 - \alpha) x_{2i})\right]^{\frac{1}{n}} \\
		& = -\left[\prod_i \alpha x_{1i} + \prod_i (1 - \alpha) x_{2i} + other\_non\_negtive\_term \right]^{\frac{1}{n}} \\
		& \leq -\left[\prod_i \alpha x_{1i} + \prod_i (1 - \alpha) x_{2i}\right]^{\frac{1}{n}} \\
		& \leq -\left[\prod_i \alpha x_{1i}\right]^{\frac{1}{n}} + \left[\prod_i (1 - \alpha) x_{2i}\right]^{\frac{1}{n}} \\
		& = \alpha f_1(\mathbf{x}_1) + (1 - \alpha) f_1(\mathbf{x}_2)
		\end{split}
		\end{equation}
	\end{exercise}

	\begin{exercise}[145.1b]
		类似a我们证明 $f_2$ 满足定义 \ref{cvx_def}. 那么我们有如下推导
		\begin{equation}
		\notag
		\begin{split}
		\alpha f_2(\mathbf{x}_1) + (1 - \alpha) f_2(\mathbf{x}_2) = & \alpha \ln(e^{x_{11}} + \ldots + e^{x_{1n}}) + (1 - \alpha) \ln(e^{x_{21}} + \ldots + e^{x_{2n}}) \\
		= & \ln \left[(e^{x_{11}} + \ldots + e^{x_{1n}})^\alpha (e^{x_{21}} + \ldots + e^{x_{2n}})^{1 - \alpha}\right] \\
		\ge & \ln(e^{\alpha x_{11} + (1 - \alpha) x_{21}} + \ldots + e^{\alpha x_{1n} + (1 - \alpha) x_{2n}}) \\
		= & f_2(\alpha \mathbf{x}_1 + (1 - \alpha) \mathbf{x}_2)
		\end{split}
		\end{equation}
	\end{exercise}

	\begin{exercise}[145.1c]
		类似a,我们有如下推导
		\begin{equation}
		\notag
		\begin{split}
		f_3(\alpha \mathbf{x}_1 + (1 - \alpha) \mathbf{x}_2) = & ((\alpha x_{11} + (1 - \alpha) x_{21})^2 + \ldots + (\alpha x_{1n} + (1 - \alpha) x_{2n})^2)^{\frac{p}{2}} \\
		\leq & ((\alpha x_{11})^2 + ((1 - \alpha) x_{21})^2 + \ldots + (\alpha x_{1n})^2 + ((1 - \alpha) x_{2n})^2)^{\frac{p}{2}} \\
		= & ((\alpha x_{11})^2 + \ldots + (\alpha x_{1n})^2 + ((1 - \alpha) x_{21})^2 + \ldots + ((1 - \alpha) x_{2n})^2)^{\frac{p}{2}} \\
		\leq & ((\alpha x_{11})^2 + \ldots + (\alpha x_{1n})^2)^{\frac{p}{2}} + (((1 - \alpha) x_{21})^2 + \ldots + ((1 - \alpha) x_{2n})^2)^{\frac{p}{2}} \\
		= & \alpha f_3(\mathbf{x}_1) + (1 - \alpha) f_3(\mathbf{x}_2)
		\end{split}
		\end{equation}
	\end{exercise}

	\begin{exercise}[145.1d]
		对于凸函数,我们知道,如果 $f(x)$ 和 $h(x)$ 都是单调递增的凸函数, 那么 $f(h(x))$ 是凸函数. 对于 $f_4(\mathbf{x}) = e^{\beta \mathbf{x}^T\mathbf{Ax}}$, 我们假设有 $f(x) = e^{\beta x}$ 并且 $h(\mathbf{x}) = \mathbf{x}^T\mathbf{Ax}$. 显然, $f(x)$ 单调递增的凸函数. 对于 $h(\mathbf{x}) = \mathbf{x}^T\mathbf{Ax}$, 因为 $\mathbf{A}$ 是对称半正定矩阵, $\mathbf{x}^T\mathbf{Ax}$ 可以写为 $\mathbf{x}^T\mathbf{Ax} = \lambda_1 p_1^2 + \ldots + \lambda_n p_n^2$. So $h(\mathbf{x})$ 也是一个凸函数. 并且 $f_4(x)$ 是个凸函数.
	\end{exercise}

	\begin{exercise}[145.2]
		考虑凸函数 $f(x) = - \ln (x)$. 使用Jasen不等式,有
		\begin{equation}
		\notag
		\begin{split}
		f(\alpha_1 x_1 + \ldots + \alpha_n x_n) \leq & \alpha_1 f(x_1) + \ldots + \alpha_n f(x_n) \Leftrightarrow \\
		- \ln(\alpha_1 x_1 + \ldots + \alpha_n x_n) \leq &  - (\alpha_1 \ln(x_1) + \ldots + \alpha_n \ln(x_n)) \Leftrightarrow \\
		x_1^{\alpha_1} \ldots x_n^{\alpha_n} \leq  & \alpha_1 x_1 + \ldots + \alpha_n x_n
		\end{split}
		\end{equation}
	\end{exercise}

	\begin{exercise}[145.3]
		由于每一个 $f_i(x)$ 都是凸函数, 我们可以将 $f(x)$ 的次梯度写成如下形式
		\begin{equation}
		\notag
		\begin{split}
		\partial f(x) = \{g = \sum_i^m g_i | f(y) \ge \sum_i^m f_i(x) + \sum_i^m<g_i, y-x>, \forall y \in C\},
		\end{split}
		\end{equation}
		其中 $g_i$ 是 $f_i(x)$的次梯度. 我们定义 $g = \sum_i^m g_i$, 因此我们可以得到
		\begin{equation}
		\partial f(x) = \sum_i^m \partial f_i(x)
		\end{equation}
	\end{exercise}

	\begin{exercise}[145.4]
		$f'(x;y) = \max_{g \in \partial f(x)} <g, y>$. 有
		\begin{equation}
		\notag
		\begin{split}
		f'(x; \alpha \mathbf{y}_1 + (1 - \alpha) \mathbf{y}_2) = & \max_{g \in \partial f(x)} <g, \alpha \mathbf{y}_1 + (1 - \alpha) \mathbf{y}_2> \\
		\leq & \alpha \max_{g \in \partial f(x)} <g, \mathbf{y}_1> + (1 - \alpha)\max_{g \in \partial f(x)} <g, \mathbf{y}_2> \\
		= & \alpha f'(x;\mathbf{y}_1) + (1 - \alpha) f'(x;\mathbf{y}_2)
		\end{split}
		\end{equation}
	\end{exercise}

	\begin{exercise}[145.5]
		l2范数的次梯度是 
		\begin{equation}
		\partial ||x||_2 = \lbrace 
		\begin{array}{cc}
		\frac{x}{||x||_2} & if \ x \neq 0 \\
		\{g | ||g||_2 \leq 1\} & if \ x = 0
		\end{array}
		\end{equation}
	\end{exercise}

	\begin{exercise}[145.6]
		为证明$f^*(y)$是凸函数,有 
		\begin{equation}
		\notag
		\begin{split}
		f^*(\alpha \mathbf{y}_1 + (1 - \alpha) \mathbf{y}_2) = & \sup_{\mathbf{x} \in C}(<\mathbf{x}, \alpha \mathbf{y}_1 + (1 - \alpha) \mathbf{y}_2> - f(\mathbf{x})) \\
		= & \sup_{\mathbf{x} \in C}(<\mathbf{x}, \alpha \mathbf{y}_1 + (1 - \alpha) \mathbf{y}_2> - (\alpha + 1 - \alpha)f(\mathbf{x})) \\
		\leq & \sup_{\mathbf{x} \in C}(<\mathbf{x}, \alpha \mathbf{y}_1 - \alpha f(\mathbf{x})) + \sup_{\mathbf{x} \in C}(<\mathbf{x}, (1 - \alpha) \mathbf{y}_2> - (1 - \alpha)f(\mathbf{x})) \\
		= & \alpha f^*(\mathbf{y}_1) + (1 - \alpha) f^*(\mathbf{y}_2)
		\end{split}
		\end{equation}
	\end{exercise}

	\begin{exercise}[145.8]
		首先,有共轭函数
		\begin{equation}
		\notag
		\begin{split}
		f^*(Y) = & \sup_{||X||_{2,1} \leq 1} \{\sum_i^n <X_i, Y_i> - ||X||_{2, 0}\} \\
		= & \sup_{||X||_{2,1} \leq 1} \{\sum_i^n ||X_i|| ||Y_i|| - ||X||_{2, 0}\} 
		\end{split}
		\end{equation}
		我们假设 $X$ 有 $k$ 列 并且l2范数等于$0$. 因此
		\begin{equation}
		\notag
		\begin{split}
		f^*(Y) = & \sup_{||X||_{2,1} \leq 1} \{\sum_{i \in ||X_i|| \neq 0} ||X_i|| ||Y_i|| - ||X||_{2, 0}\} \\
		= & \sup_{||X||_{2,1} \leq 1} \{\sum_{i \in ||X_i|| \neq 0}||Y_i|| - k\}
		\end{split}
		\end{equation}
		其中 $i$ 是满足 $||X_i|| \neq 0$ 的下标, $||Y_i||$ 也是下标. 那么 $f^*(Y)$ 可以按如下方式计算
		\begin{equation}
		\notag
		\begin{split}
		f^*(Y) = \sum_{i \in ||X_i|| \neq 0}(||Y_i|| - 1)
		\end{split}
		\end{equation}

		我们按照如下方式计算 $f^*(Y)$ 的共轭函数
		\begin{equation}
		\notag
		\begin{split}
		f^{**}(Z) = & \sup_Y \{\sum_i^n <Y_i, Z_i> - \sum_{i \in ||X_i|| \neq 0}(||Y_i|| - 1) \} \\
		= & \sup_Y \{\sum_i^n ||Y_i|| ||Z_i|| - \sum_{i \in ||X_i|| \neq 0}(||Y_i|| - 1)\} 
		\end{split}
		\end{equation}
		其中 $||Z_i||$ 按如下方式取到最大值
		\begin{equation}
		\notag
		\begin{split}
		f^{**}(Z) = \sum_{i \in ||X_i|| \neq 0} ||Z_i||
		\end{split}
		\end{equation}
	\end{exercise}

	\begin{exercise}[146.1]
		我们知道 $W = \alpha \alpha^T$, 所以 $Tr(W) = \lambda_1 = 1$. 因为 $\lambda_1 \ge 0$, 我摸可以知道 $W$ 是半正定矩阵.

		对于 $\mathbf{X} = \mathbf{0}$的情形, $W = uv^T$的核范数可以按照下式计算 $||W||_* = Tr(W^TW) = Tr(vu^Tuv^T) \leq 1$.

	\end{exercise}

	\begin{exercise}[146.2]
		l2范数的凸包是
		\begin{equation}
		\notag
		\begin{split}
		E_c(x) = \inf_w \{||w||_2 + \frac{1}{2c}||w - x||^2\}
		\end{split}
		\end{equation}

		l2范数的proximal mapping是 
		\begin{equation}
		\notag
		\begin{split}
		P_c(x) = \arg \min_w \{||w||_2 + \frac{1}{2c}||w - x||^2\}
		\end{split}
		\end{equation}
		因此有
		\begin{equation}
		\notag
		\begin{split}
		P_c(x) = \{
		\begin{array}{cc}
		x - \frac{cx}{||x||_2} & if \ c < ||x||_2 \\
		0 & if \ c \ge ||x||_2
		\end{array}
		\end{split}
		\end{equation}

	\end{exercise}

	\begin{exercise}[146.3]
		我们知道 $u = P_cf(x) \Leftrightarrow x - u \in \partial f(u)$, 那么有如下证明
		\begin{equation}
		\notag
		\begin{split}
		& x - \frac{1}{\lambda}[P_c(\lambda^2 g)(\lambda x + a) - a] \in \lambda \partial g(\lambda u + a) \Leftrightarrow \\
		& \lambda x + a - P_c(\lambda^2 g)(\lambda x + a) \in \lambda^2 \partial g(P_c(\lambda^2 g)(\lambda x + a)) \Leftrightarrow \\
		& x' - u' \in \partial (\lambda^2 g)(u')
		\end{split}
		\end{equation}
	\end{exercise}

	\begin{exercise}[146.4]
		类似与3,我们有如下证明
		\begin{equation}
		\notag
		\begin{split}
		& x - {\lambda}[P_c(\lambda^{-1} g)(x / \lambda)] \in \partial g(u / \lambda) \Leftrightarrow \\
		& x / \lambda - P_c(\lambda^{-1} g)(x / \lambda) \in \partial g(P_c(\lambda^{-1} g)(x / \lambda)) \Leftrightarrow \\
		& x' - u' \in \partial (\lambda^{-1} g)(u')
		\end{split}
		\end{equation}
	\end{exercise}
\end{document}