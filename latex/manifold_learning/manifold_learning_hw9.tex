%!TEX program = xelatex
\documentclass[a4paper, UTF8]{ctexart}
\usepackage{ctex}
\usepackage{amsmath}
\usepackage{amsthm}
\usepackage{multirow}
\usepackage{amssymb}
\usepackage{graphicx}
\usepackage{geometry}
\usepackage{bm}
\usepackage{subfigure}
\usepackage{float}
\usepackage{mathrsfs}
\renewcommand\thesection{\arabic{section}}
\newtheorem*{exercise}{\textbf{习题}}
\newtheorem*{theorem}{Theorem}
\title{Manifole Learning Homework 9}
\date{\today}
\author{安捷 1601210097}
\begin{document}
\maketitle
	\begin{exercise}[155]
		\begin{equation}
			\mu \left( A \right) = \max\limits_{i,j} \langle \phi_i^T, \psi_j \rangle = \max\limits_{i,j}DFT_{i,j} = \frac{1}{\sqrt{n}}
		\end{equation}
	\end{exercise}
	\begin{exercise}[160]
		首先,原矩阵rank=2,且矩阵$A$中的任意两列都是线性无关的,故若$x$中有两个元素为$0$,则$Ax = 0$只有零解,即$x$中至少要有三个元素非$0$,又由下一题结论,$spark\left(A\right)=3$
	\end{exercise}
	\begin{exercise}[161]
		首先,假设原矩阵rank=r,当取对应与原矩阵$A$的r个极大线性无关列的$x$元素不为$0$时,$x$只有零解,故只有当$x$的$r+1$个元素非$0$时,才有非零解,并且,此时一定有非零解,故存在
		\begin{equation}
			spark \left( A \right) = rank \left( A \right) + 1
		\end{equation}
		又显然上述等式可以取小于号,故有
		\begin{equation}
			spark \left( A \right) \le rank \left( A \right) + 1
		\end{equation}
	\end{exercise}
\end{document}