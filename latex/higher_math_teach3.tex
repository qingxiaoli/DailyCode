\documentclass[a4paper, UTF8]{ctexart}
\usepackage{ctex}
\usepackage{amsmath}
\usepackage{multirow}
\usepackage{amssymb}
\usepackage{graphicx}
\usepackage{geometry}
\usepackage{bm}
\usepackage{subfigure}
\usepackage{float}

\renewcommand\thesection{\arabic{section}}
\newtheorem{exercise}{习题}

\title{高数第三次习题课讲义}
\date{2017-03-07}

\begin{document}
\maketitle
\section{作业问题}
\begin{enumerate}
	\item 关于积分函数与定积分结果之间关系的讨论
	\item 关于连续函数性质的讨论
	\item P5 4
	\item P22 5
\end{enumerate}

\section{课堂补充}
\label{sec:课堂补充}
\begin{enumerate}
	\item 关于Jacobi行列式的再讨论
	\item 常用坐标变换的小总结
	\item 积分区域对称情况下三重积分奇偶性的讨论
	\item 二重积分与三重积分在积分区域对称情况下奇偶性相关问题的再讨论
\end{enumerate}

\section{习题讲解}
\label{sec:习题讲解}
\begin{exercise}
	$I = \iint\limits_D \left( x^2 + y^2 \right)\mathrm{d}x \mathrm{d}y$,其中 $D$为椭圆: $\frac{^2}{a^2} + \frac{y^2}{b^2}\le 1$;
\end{exercise}

\begin{exercise}
	$I = \iint\limits_D xy \mathrm{d}x \mathrm{d}y$,其中 $D$由曲线 $x^2=y,x^2=4y,x=y^2$及 $4x=y^2$所围;
\end{exercise}

\begin{exercise}
	求积分
	\begin{equation*}
		\iiint\limits_\Omega \frac{\mathrm{d}v}{\left( x+y+z+1 \right)^3}
	\end{equation*}
	其中 $\Omega$为由三个坐标平面和平面 $x+y+z=1$所围的区域;
\end{exercise}
 \begin{exercise}
 	求 $\iiint\limits_D xy \mathrm{d}v$,其中 $\Omega$是由曲面 $z=xy,z=0,x+y=1$所围成的区域;
 \end{exercise}

 \begin{exercise}
 	求 $\iiint\limits_\Omega \left( z+z^2 \right)\mathrm{d}v$其中 $\Omega$为单位球: $x^2+y^2+z^2\le1$;
 \end{exercise}
\end{document}
