\documentclass[a4paper, UTF8]{ctexart}
\usepackage{ctex}
\usepackage{amsmath}
\usepackage{amsthm}
\usepackage{multirow}
\usepackage{amssymb}
\usepackage{graphicx}
\usepackage{geometry}
\usepackage{bm}
\usepackage{subfigure}
\usepackage{float}
\usepackage{mathrsfs}

\renewcommand\thesection{\arabic{section}}
\newtheorem*{exercise}{\textbf{习题}}

\title{Manifold Learning Homework 2}
\date{2017-03-01}
\author{安捷 1601210097}

\begin{document}
\maketitle
\begin{exercise}[24]
  \begin{proof}[\textbf{Proof} 1]
    \begin{equation*}
      \lVert \cdot \rVert_p^* = \lVert \cdot \rVert_q
    \end{equation*}
    \begin{equation*}
      \langle \mathbf{x}, \mathbf{y} \rangle
    \end{equation*}
  \end{proof}
\end{exercise}
\begin{exercise}[28]
	\begin{proof}[\textbf{Proof} (2.20)]
		\begin{equation*}
			\frac{\partial \mathbf{XY}}{\partial t} = \frac{\partial \mathbf{X \left(t\right)}\mathbf{Y \left(t\right)}}{\partial t}
		\end{equation*}
		\begin{equation*}
			= \frac{ \partial \left( \mathbf{X \left( t \right) Y \left( t \right)} \right)_{ij} }{\partial t}
		\end{equation*}
		其中$\partial \left( \mathbf{X \left( t \right) Y \left( t \right)} \right)_{ij}$表示两矩阵乘积的第 $i,j$个元素,下面考虑$\partial \left( \mathbf{X \left( t \right) Y \left( t \right)} \right)_{ij}$关于 $t$的偏导数
		\begin{equation*}
			\frac{\partial \mathbf{X \left( t \right) Y \left( t \right)}_{ij} }{\partial t} = \frac{\partial \Sigma_{k=1}^n \mathbf{X}_{ik} \left( t \right)\mathbf{Y}_{kj} \left( t \right)}{\partial t}
		\end{equation*}
		\begin{equation*}
			= \Sigma_{k=1}^n \frac{\partial \left( \mathbf{X}_{ik}\left( t \right)\mathbf{Y}_{kj}\left( t \right) \right)}{\partial t}
		\end{equation*}
		\begin{equation*}
			= \Sigma_{k=1}^n\left( \frac{\partial\mathbf{X}_{ik}\left(t\right)}{\partial t}\mathbf{Y}_{kj}\left( t \right) + \frac{\partial \mathbf{Y}_{kj}\left( t \right)}{\partial t}\mathbf{X}_{ik}\left( t \right)\right)
		\end{equation*}
		\begin{equation*}
			= \Sigma_{k=1}^n \left( \frac{\partial \mathbf{X}_{ik} \left( t \right) }{\partial t}\mathbf{Y}_{kj}\left( t \right) \right) + \Sigma_{k=1}^n \left( \frac{\partial \mathbf{Y}_{kj}\left( t \right)}{\partial t} \mathbf{X}_{ik}\left( t \right)\right)
		\end{equation*}
		\begin{equation*}
			= \left( \frac{\mathbf{X}\left( t \right)}{\left( t \right)}\mathbf{Y} \left( t \right)\right)_{ij} + \left( \frac{\partial \mathbf{Y}\left( t \right)}{\partial t }\mathbf{X}\left( t \right)\right)_{ij}
		\end{equation*}
		\begin{equation*}
			\left( \frac{\mathbf{X}\left( t \right)}{\partial t}\mathbf{Y}\left( t \right) + \frac{\partial \mathbf{Y}\left( t \right)}{\partial t}\mathbf{X}\left( t \right)\right)_{ij}
		\end{equation*}
		所以有
		\begin{equation*}
			\frac{\partial\mathbf{XY}}{\partial t} = \frac{\partial \mathbf{X}}{\partial t} \mathbf{Y} + \frac{\partial \mathbf{Y}}{\partial t} \mathbf{X}
		\end{equation*}
	\end{proof}
	\begin{proof}[\textbf{Proof} (2.25)]
		\begin{equation*}
			\frac{\partial \left(\mathbf{a}^T\mathbf{x}\right)}{\partial \mathbf{x}} = \mathbf{a}
		\end{equation*}
		\begin{equation*}
			\frac{\partial \left( \mathbf{a}^T \mathbf{x} \right)}{\partial \mathbf{x}_i}= \frac{\partial \left( \Sigma_{i=1}^n \mathbf{a}_i \mathbf{x}_i \right)}{\partial \mathbf{x}_i} = \mathbf{a}_i
		\end{equation*}
		所以有
		\begin{equation*}
			\frac{\partial \left( \mathbf{a}^T \mathbf{x} \right)}{\partial \mathbf{x}} = \mathbf{a}
		\end{equation*}
	\end{proof}
	\begin{proof}[\textbf{Proof} (2.26)]
		\begin{equation*}
			\frac{\partial \left(\mathbf{x}^T \mathbf{A} \mathbf{x}\right)}{\partial \mathbf{x}} = \left( \mathbf{A} + \mathbf{A}^T \right) \mathbf{x}
		\end{equation*}
		考虑导数的第 $k$ 个元素,有
		\begin{equation*}
			\frac{\partial \left( \mathbf{x}^T \mathbf{A}\mathbf{x} \right)}{\partial \mathbf{x}_k} = \frac{\partial \left( \Sigma_{j=1}^n \left( \Sigma_{k=1}^n \mathbf{x}_k \mathbf{A}_{kj} \right) \mathbf{x}_j \right)}{\partial \mathbf{x}_i}
		\end{equation*}
		\begin{equation*}
			= \frac{\partial \left( \Sigma_{k\neq i} \mathbf{x}_k \mathbf{A}_{ki} \mathbf{x}_i + \Sigma_{j \neq i} \mathbf{A}_{ij} \mathbf{x}_j \mathbf{x}_i + \mathbf{A}_{ii}\mathbf{x}_i^2 \right)}{\partial \mathbf{x}_i}
		\end{equation*}
		\begin{equation*}
			= \Sigma_{k\neq i}\mathbf{A}_{ki} \mathbf{x}_k + \Sigma_{j\neq i}\mathbf{A}_{ij} \mathbf{x}_j + 2 \mathbf{A}_{ii} \mathbf{x}_i
		\end{equation*}
		\begin{equation*}
			=\Sigma_{t=1}^n \mathbf{A}_{ti} \mathbf{x}_t + \Sigma_{t=1}^n \mathbf{A}_{it} \mathbf{x}_t
		\end{equation*}
		\begin{equation*}
			= \left( \mathbf{A} + \mathbf{A}^T \right)_i \mathbf{x}
		\end{equation*}
		所以有
		\begin{equation*}
			\frac{\partial \left( \mathbf{x}^T \mathbf{A} \mathbf{x} \right)}{\partial \mathbf{x}} = \left( \mathbf{A} + \mathbf{A}^T \right) \mathbf{x}
		\end{equation*}
	\end{proof}
	\begin{proof}[\textbf{Proof} $\frac{\partial \mathbf{X}}{\partial \mathbf{X}}$]
		\begin{equation*}
			\frac{\partial \mathbf{X}}{\partial \mathbf{X}}
		\end{equation*}
		首先看
		\begin{equation*}
			\frac{\partial \mathbf{X}}{\partial \mathbf{X}_{ij}}
		\end{equation*}
		有
		\begin{equation*}
			\frac{\partial \mathbf{X}}{\partial \mathbf{X}_{ij}} = \mathbf{e}_i \mathbf{e}_j^T
		\end{equation*}
		其中为只有第 $i$行为 $1$,其余行皆为 $0$的向量,则有
		\begin{equation*}
			\frac{\partial \mathbf{X}}{\partial \mathbf{X}} = \left(
			\begin{array}{c}
				\mathbf{e}_1 \\
				\mathbf{e}_2 \\
				\mathbf{e}_3 \\
				\vdots       \\
				\mathbf{e}_n \\
			\end{array}
			\right) \left(
			\mathbf{e}_1^T, \mathbf{e}_2^T, \mathbf{e}_3^T, \dots, \mathbf{e}_n^T
			\right)
		\end{equation*}
	\end{proof}
\end{exercise}

\begin{exercise}[29]
	\begin{equation*}
		\langle \mathcal{A}^* \left( \mathbf{x} \right), \mathbf{y} \rangle = \langle \mathbf{x}, \mathcal{A}\left( y \right) \rangle
	\end{equation*}
	所以有
	\begin{equation*}
		\langle \mathcal{A}^* \left( x \right), \mathbf{Y} \rangle = \left( Y_{11} + Y_{12} - Y_{31} + 2Y_{33} \right) x
	\end{equation*}
	\begin{equation*}
		= \langle \left(
		\begin{array}{ccc}
			x  & x & 0  \\
			0  & 0 & 0  \\
			-x & 0 & 2x \\
		\end{array}
		\right) ,\mathbf{Y} \rangle
	\end{equation*}
	因此
	\begin{equation*}
		\mathcal{A}^*\left( x \right) = \left(
		\begin{array}{ccc}
			x  & x & 0  \\
			0  & 0 & 0  \\
			-x & 0 & 2x \\
		\end{array} \right)
	\end{equation*}
\end{exercise}
\end{document}
