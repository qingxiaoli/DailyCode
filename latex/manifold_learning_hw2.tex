\documentclass[a4paper, UTF8]{ctexart}
\usepackage{ctex}
\usepackage{amsmath}
\usepackage{amsthm}
\usepackage{multirow}
\usepackage{amssymb}
\usepackage{graphicx}
\usepackage{geometry}
\usepackage{bm}
\usepackage{subfigure}
\usepackage{float}

\renewcommand\thesection{\arabic{section}}
\newtheorem*{exercise}{\textbf{习题}}

\title{Manifold Learning Homework 2}
\date{2017-03-01}
\author{安捷 1601210097}

\begin{document}
\maketitle
  \begin{exercise}[25]
    \begin{proof}[\textbf{proof} (2.20)]
      \begin{equation*}
        \frac{\partial \mathbf{XY}}{\partial t} = \frac{\partial \mathbf{X \left(t\right)}\mathbf{Y \left(t\right)}}{\partial t}
      \end{equation*}
      \begin{equation*}
        = \frac{ \partial \left( \mathbf{X \left( t \right) Y \left( t \right)} \right)_{ij} }{\partial t}
      \end{equation*}
      其中$\partial \left( \mathbf{X \left( t \right) Y \left( t \right)} \right)_{ij}$表示两矩阵乘积的第 $i,j$个元素,下面考虑$\partial \left( \mathbf{X \left( t \right) Y \left( t \right)} \right)_{ij}$关于 $t$的偏导数
      \begin{equation*}
        \frac{\partial \mathbf{X \left( t \right) Y \left( t \right)}_{ij} }{\partial t} = \frac{\partial \Sigma_{k=1}^n \mathbf{X}_{ik} \left( t \right)\mathbf{Y}_{kj} \left( t \right)}{\partial t}
      \end{equation*}
      \begin{equation*}
        = \Sigma_{k=1}^n \frac{\partial \left( \mathbf{X}_{ik}\left( t \right)\mathbf{Y}_{kj}\left( t \right) \right)}{\partial t}
      \end{equation*}
      \begin{equation*}
        = \Sigma_{k=1}^n\left( \frac{\partial\mathbf{X}_{ik}\left(t\right)}{\partial t}\mathbf{Y}_{kj}\left( t \right) + \frac{\partial \mathbf{Y}_{kj}\left( t \right)}{\partial t}\mathbf{X}_{ik}\left( t \right)\right)
      \end{equation*}
      \begin{equation*}
        = \Sigma_{k=1}^n \left( \frac{\partial \mathbf{X}_{ik} \left( t \right) }{\partial t}\mathbf{Y}_{kj}\left( t \right) \right) + \Sigma_{k=1}^n \left( \frac{\partial \mathbf{Y}_{kj}\left( t \right)}{\partial t} \mathbf{X}_{ik}\left( t \right)\right)
      \end{equation*}
      \begin{equation*}
        = \left( \frac{\mathbf{X}\left( t \right)}{\left( t \right)}\mathbf{Y} \left( t \right)\right)_{ij} + \left( \frac{\partial \mathbf{Y}\left( t \right)}{\partial t }\mathbf{X}\left( t \right)\right)_{ij}
      \end{equation*}
      \begin{equation*}
        \left( \frac{\mathbf{X}\left( t \right)}{\partial t}\mathbf{Y}\left( t \right) + \frac{\partial \mathbf{Y}\left( t \right)}{\partial t}\mathbf{X}\left( t \right)\right)_{ij}
      \end{equation*}
      所以有
      \begin{equation*}
        \frac{\partial\mathbf{XY}}{\partial t} = \frac{\partial \mathbf{X}}{\partial t} \mathbf{Y} + \frac{\partial \mathbf{Y}}{\partial t} \mathbf{X}
      \end{equation*}
    \end{proof}
    \begin{proof}[\textbf{proof} (2.25)]
      \begin{equation*}
        \frac{\partial \left(\mathbf{a}^T\mathbf{x}\right)}{\partial \mathbf{x}} = \mathbf{a}
      \end{equation*}
      \begin{equation*}
        \frac{\partial \left( \mathbf{a}^T \mathbf{x} \right)}{\partial \mathbf{x}_i}= \frac{\partial \left( \Sigma_{i=1}^n \mathbf{a}_i \mathbf{x}_i \right)}{\partial \mathbf{x}_i} = \mathbf{a}_i
      \end{equation*}
      所以有
      \begin{equation*}
        \frac{\partial \left( \mathbf{a}^T \mathbf{x} \right)}{\partial \mathbf{x}} = \mathbf{a}
      \end{equation*}
    \end{proof}
    \begin{proof}[\textbf{proof} (2.26)]
      \begin{equation*}
        \frac{\partial \left(\mathbf{x}^T \mathbf{A} \mathbf{x}\right)}{\partial \mathbf{x}} = \left( \mathbf{A} + \mathbf{A}^T \right) \mathbf{x}
      \end{equation*}
      考虑导数的第 $k$ 个元素,有
      \begin{equation*}
        \frac{\partial \left( \mathbf{x}^T \mathbf{A}\mathbf{x} \right)}{\partial \mathbf{x}_k} = \frac{\partial \left( \Sigma_{j=1}^n \left( \Sigma_{k=1}^n \mathbf{x}_k \mathbf{A}_{kj} \right) \mathbf{x}_j \right)}{\partial \mathbf{x}_i}
      \end{equation*}
      \begin{equation*}
        = \frac{\partial \left( \Sigma_{k\neq i} \mathbf{x}_k \mathbf{A}_{ki} + \Sigma_{j \neq i} \mathbf{A}_{ij} \mathbf{x}_j + \mathbf{A}_{ii}\mathbf{x}_i^2 \right)}{\partial \mathbf{x}_i}
      \end{equation*}
    \end{proof}
  \end{exercise}
\end{document}
