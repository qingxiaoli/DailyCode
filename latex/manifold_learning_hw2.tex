\documentclass[a4paper, UTF8]{ctexart}
\usepackage{ctex}
\usepackage{amsmath}
\usepackage{amsthm}
\usepackage{multirow}
\usepackage{amssymb}
\usepackage{graphicx}
\usepackage{geometry}
\usepackage{bm}
\usepackage{subfigure}
\usepackage{float}
\usepackage{mathrsfs}

\renewcommand\thesection{\arabic{section}}
\newtheorem*{exercise}{\textbf{习题}}
\newtheorem*{theorem}{Theorem}

\title{Manifold Learning Homework 2}
\date{2017-03-01}
\author{安捷 1601210097}

\begin{document}
\maketitle
\begin{exercise}[24]
  \begin{proof}[Proof 1]
    \begin{equation*}
      \lVert \cdot \rVert_p^* = \lVert \cdot \rVert_q
    \end{equation*}
    由于Holder inequality
    \begin{theorem}
      对于任意的 $p > 1, q>1$,如果满足
      \begin{equation*}
        \frac{1}{p} + \frac{1}{q} = 1
      \end{equation*}
      那么有如下关系成立
      \begin{equation*}
        \sum_{k=1}^n |a_k b_k| \le \left( \sum_{k=1}^n |a_k|^p \right)^{\frac{1}{p}}\left( \sum_{k=1}^n |b_k|^q \right)^{\frac{1}{q}}
      \end{equation*}
    \end{theorem}
  由于$\lVert \mathbf{y}\rVert_p \le 1$,因此有 $\langle \mathbf{x}, \mathbf{y}\rangle \le \lVert \mathbf{x}_q \rVert$,因此有 $\lVert \mathbf{x} \rVert_p^* = \lVert \mathbf{x} \rVert_q$
  \end{proof}
  \begin{proof}[Proof 2]
    \begin{equation*}
      \langle \mathbf{x}, \mathbf{y} \rangle = \rm{tr} \left( \mathbf{x}^T \mathbf{y} \right) = \rm{tr} \left( \mathbf{B}^{-1}\mathbf{x}^T \mathbf{y} \mathbf{B} \right)
    \end{equation*}
    其中,有
    \begin{equation*}
      \mathbf{B}^{-1} \mathbf{y} \mathbf{B}=\mathbf{V}
    \end{equation*}
    $\mathbf{V}$为对角矩阵且对角线元素为矩阵 $\mathbf{y}$的特征值
    \begin{equation*}
      \rm{tr}\left( \mathbf{B}^{-1} \mathbf{x}^T \mathbf{y} \mathbf{B} \right) = \rm{tr} \left( \mathbf{x}^T \mathbf{V} \right)
    \end{equation*}
    又有
    \begin{equation*}
      \lVert \mathbf{y} \rVert_2 \le 1
    \end{equation*}
    所以 $V$的元素的绝对值都小于等于 $1$,因此有
    \begin{equation*}
      \rm{tr} \left( \mathbf{x}^T \mathbf{V} \right) \le \sum_{k=1}^n |\lambda_i| = \lVert \mathbf{x} \rVert_*
    \end{equation*}
  \end{proof}
  \begin{proof}[Proof 3]
    由 $p=2,q=2$时的Holder inequality,即Cauthy inequality
    \begin{equation*}
      \langle \mathbf{x}, \mathbf{y} \rangle = \sum_{ij}\mathbf{x}_{ij}\mathbf{y}{ij} \le \left( \sum_{ij} |\mathbf{x}_{ij}|^2 \right)^{\frac{1}{2}} + \left( \sum_{ij} |\mathbf{y}_{ij}|^2 \right)^{\frac{1}{2}} \le \lVert \mathbf{x} \rVert_F
    \end{equation*}
    所以有
    \begin{equation*}
      \lVert \mathbf{x} \rVert_F^* = \lVert \mathbf{x} \rVert_F
    \end{equation*}
  \end{proof}
\end{exercise}
\begin{exercise}[28]
	\begin{proof}[Proof (2.20)]
		\begin{equation*}
			\frac{\partial \mathbf{XY}}{\partial t} = \frac{\partial \mathbf{X \left(t\right)}\mathbf{Y \left(t\right)}}{\partial t}
		\end{equation*}
		\begin{equation*}
			= \frac{ \partial \left( \mathbf{X \left( t \right) Y \left( t \right)} \right)_{ij} }{\partial t}
		\end{equation*}
		其中$\partial \left( \mathbf{X \left( t \right) Y \left( t \right)} \right)_{ij}$表示两矩阵乘积的第 $i,j$个元素,下面考虑$\partial \left( \mathbf{X \left( t \right) Y \left( t \right)} \right)_{ij}$关于 $t$的偏导数
		\begin{equation*}
			\frac{\partial \mathbf{X \left( t \right) Y \left( t \right)}_{ij} }{\partial t} = \frac{\partial \sum_{k=1}^n \mathbf{X}_{ik} \left( t \right)\mathbf{Y}_{kj} \left( t \right)}{\partial t}
		\end{equation*}
		\begin{equation*}
			= \sum_{k=1}^n \frac{\partial \left( \mathbf{X}_{ik}\left( t \right)\mathbf{Y}_{kj}\left( t \right) \right)}{\partial t}
		\end{equation*}
		\begin{equation*}
			= \sum_{k=1}^n\left( \frac{\partial\mathbf{X}_{ik}\left(t\right)}{\partial t}\mathbf{Y}_{kj}\left( t \right) + \frac{\partial \mathbf{Y}_{kj}\left( t \right)}{\partial t}\mathbf{X}_{ik}\left( t \right)\right)
		\end{equation*}
		\begin{equation*}
			= \sum_{k=1}^n \left( \frac{\partial \mathbf{X}_{ik} \left( t \right) }{\partial t}\mathbf{Y}_{kj}\left( t \right) \right) + \sum_{k=1}^n \left( \frac{\partial \mathbf{Y}_{kj}\left( t \right)}{\partial t} \mathbf{X}_{ik}\left( t \right)\right)
		\end{equation*}
		\begin{equation*}
			= \left( \frac{\mathbf{X}\left( t \right)}{\left( t \right)}\mathbf{Y} \left( t \right)\right)_{ij} + \left( \frac{\partial \mathbf{Y}\left( t \right)}{\partial t }\mathbf{X}\left( t \right)\right)_{ij}
		\end{equation*}
		\begin{equation*}
			\left( \frac{\mathbf{X}\left( t \right)}{\partial t}\mathbf{Y}\left( t \right) + \frac{\partial \mathbf{Y}\left( t \right)}{\partial t}\mathbf{X}\left( t \right)\right)_{ij}
		\end{equation*}
		所以有
		\begin{equation*}
			\frac{\partial\mathbf{XY}}{\partial t} = \frac{\partial \mathbf{X}}{\partial t} \mathbf{Y} + \frac{\partial \mathbf{Y}}{\partial t} \mathbf{X}
		\end{equation*}
	\end{proof}
	\begin{proof}[Proof (2.25)]
		\begin{equation*}
			\frac{\partial \left(\mathbf{a}^T\mathbf{x}\right)}{\partial \mathbf{x}} = \mathbf{a}
		\end{equation*}
		\begin{equation*}
			\frac{\partial \left( \mathbf{a}^T \mathbf{x} \right)}{\partial \mathbf{x}_i}= \frac{\partial \left( \sum_{i=1}^n \mathbf{a}_i \mathbf{x}_i \right)}{\partial \mathbf{x}_i} = \mathbf{a}_i
		\end{equation*}
		所以有
		\begin{equation*}
			\frac{\partial \left( \mathbf{a}^T \mathbf{x} \right)}{\partial \mathbf{x}} = \mathbf{a}
		\end{equation*}
	\end{proof}
	\begin{proof}[Proof (2.26)]
		\begin{equation*}
			\frac{\partial \left(\mathbf{x}^T \mathbf{A} \mathbf{x}\right)}{\partial \mathbf{x}} = \left( \mathbf{A} + \mathbf{A}^T \right) \mathbf{x}
		\end{equation*}
		考虑导数的第 $k$ 个元素,有
		\begin{equation*}
			\frac{\partial \left( \mathbf{x}^T \mathbf{A}\mathbf{x} \right)}{\partial \mathbf{x}_k} = \frac{\partial \left( \sum_{j=1}^n \left( \sum_{k=1}^n \mathbf{x}_k \mathbf{A}_{kj} \right) \mathbf{x}_j \right)}{\partial \mathbf{x}_i}
		\end{equation*}
		\begin{equation*}
			= \frac{\partial \left( \sum_{k\neq i} \mathbf{x}_k \mathbf{A}_{ki} \mathbf{x}_i + \sum_{j \neq i} \mathbf{A}_{ij} \mathbf{x}_j \mathbf{x}_i + \mathbf{A}_{ii}\mathbf{x}_i^2 \right)}{\partial \mathbf{x}_i}
		\end{equation*}
		\begin{equation*}
			= \sum_{k\neq i}\mathbf{A}_{ki} \mathbf{x}_k + \sum_{j\neq i}\mathbf{A}_{ij} \mathbf{x}_j + 2 \mathbf{A}_{ii} \mathbf{x}_i
		\end{equation*}
		\begin{equation*}
			=\sum_{t=1}^n \mathbf{A}_{ti} \mathbf{x}_t + \sum_{t=1}^n \mathbf{A}_{it} \mathbf{x}_t
		\end{equation*}
		\begin{equation*}
			= \left( \mathbf{A} + \mathbf{A}^T \right)_i \mathbf{x}
		\end{equation*}
		所以有
		\begin{equation*}
			\frac{\partial \left( \mathbf{x}^T \mathbf{A} \mathbf{x} \right)}{\partial \mathbf{x}} = \left( \mathbf{A} + \mathbf{A}^T \right) \mathbf{x}
		\end{equation*}
	\end{proof}
	\begin{proof}[Proof $\frac{\partial \mathbf{X}}{\partial \mathbf{X}}$]
		\begin{equation*}
			\frac{\partial \mathbf{X}}{\partial \mathbf{X}}
		\end{equation*}
		首先看
		\begin{equation*}
			\frac{\partial \mathbf{X}}{\partial \mathbf{X}_{ij}}
		\end{equation*}
		有
		\begin{equation*}
			\frac{\partial \mathbf{X}}{\partial \mathbf{X}_{ij}} = \mathbf{e}_i \mathbf{e}_j^T
		\end{equation*}
		其中为只有第 $i$行为 $1$,其余行皆为 $0$的向量,则有
		\begin{equation*}
			\frac{\partial \mathbf{X}}{\partial \mathbf{X}} = \left(
			\begin{array}{c}
				\mathbf{e}_1 \\
				\mathbf{e}_2 \\
				\mathbf{e}_3 \\
				\vdots       \\
				\mathbf{e}_n \\
			\end{array}
			\right) \left(
			\mathbf{e}_1^T, \mathbf{e}_2^T, \mathbf{e}_3^T, \dots, \mathbf{e}_n^T
			\right)
		\end{equation*}
	\end{proof}
\end{exercise}

\begin{exercise}[29]
	\begin{equation*}
		\langle \mathcal{A}^* \left( \mathbf{x} \right), \mathbf{y} \rangle = \langle \mathbf{x}, \mathcal{A}\left( y \right) \rangle
	\end{equation*}
	所以有
	\begin{equation*}
		\langle \mathcal{A}^* \left( x \right), \mathbf{Y} \rangle = \left( Y_{11} + Y_{12} - Y_{31} + 2Y_{33} \right) x
	\end{equation*}
	\begin{equation*}
		= \langle \left(
		\begin{array}{ccc}
			x  & x & 0  \\
			0  & 0 & 0  \\
			-x & 0 & 2x \\
		\end{array}
		\right) ,\mathbf{Y} \rangle
	\end{equation*}
	因此
	\begin{equation*}
		\mathcal{A}^*\left( x \right) = \left(
		\begin{array}{ccc}
			x  & x & 0  \\
			0  & 0 & 0  \\
			-x & 0 & 2x \\
		\end{array} \right)
	\end{equation*}
\end{exercise}
\begin{exercise}[33]
  \begin{proof}
    首先,由于有
    \begin{equation*}
      \rm{tr}\left( \mathbf{Y}\mathbf{K}\mathbf{Y}^T \right)=\rm{tr}\left( \mathbf{Y}\left( \mathbf{K} + a\mathbf{I}  \right)\mathbf{Y}^T \right)-a
    \end{equation*}
    因此,可以通过选择一个足够大的 $a$使得矩阵 $\mathbf{K}$ 变为正定矩阵,因此,这里不妨设矩阵 $\mathbf{K}$ 为正定矩阵,这样做只会使得目标函数增加一个常数,不会对优化的求解产生任何影响。
    \begin{equation*}
      \rm{tr}\left( \mathbf{Y} \mathbf{K} \mathbf{Y}^T \right)=\rm{tr}\left( \mathbf{Y}^T \mathbf{Y} \mathbf{K} \right)
    \end{equation*}
    由于 $\sigma \left( \mathbf{Y}^T \mathbf{Y} \right)=\sigma \left( \mathbf{Y} \mathbf{Y}^T \right) = 1$,由von Neumann迹定理,有
    \begin{equation*}
      \sum_{i=1}^m \sigma_{n-i+1}\left( \mathbf{K} \right)\le \rm{tr}\left( \mathbf{Y}^T \mathbf{Y} \mathbf{K} \right) \le \sum_{i=1}^m \sigma_{i} \left( \mathbf{K} \right)
    \end{equation*}
    其中, $m$为 $\mathbf{Y}$ 的行数,下面说明什么时候可以取到最值,当 $\mathbf{Y}^T$ 取为上述定理中最大的 $m$个奇异值,即最大特征值对应的特征向量时,有
    \begin{equation*}
      \rm{tr}\left( \mathbf{Y}\mathbf{K}\mathbf{Y}^T \right) = \rm{tr}\left( \mathbf{Y}\mathbf{Y}^T \mathbf{V} \right)
    \end{equation*}
    其中, $\mathbf{V}$是矩阵 $\mathbf{K}$的前 $m$个特征值组成的用零扩展的对角矩阵,即上半部分为对角矩阵,矩阵维数为 $m$,下半部分为零矩阵,此时,可以取到最大值。对应的,当取最小特征值对应的特征向量为 $\mathbf{Y}$时,可以取到最小值。我尝试用拉格朗日乘子法求解,由于约束条件取范数之后难以求导,因此我没有办法用拉格朗日乘子法证明上述结论。
  \end{proof}
\end{exercise}
\end{document}
