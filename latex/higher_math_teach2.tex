\documentclass[a4paper, UTF8]{ctexart}
\usepackage{ctex}
\usepackage{amsmath}
\usepackage{multirow}
\usepackage{amssymb}
\usepackage{graphicx}
\usepackage{geometry}
\usepackage{bm}
\usepackage{subfigure}
\usepackage{float}

\renewcommand\thesection{\arabic{section}}
\newtheorem{exercise}{习题}

\title{高数第二次习题课讲义} 
\date{2017-02-27}

\begin{document}
\maketitle
\section{课堂补充}
\label{sec:课堂补充}
\begin{enumerate}
	\item 积分区间对称情况下二重积分函数奇偶性质的讨论 P9
	\item 微元法证明二重积分的极坐标公式
	\item 极坐标变换怎么才能不出错
\end{enumerate}	
\section{习题}
\label{sec:习题}
\begin{exercise}
	$\iint_D \frac{x}{x^2 + y^2}dxdy$,其中,$D$是由抛物线$y=\frac{x^2}{2}$和直线$y=x$所围成;
\end{exercise}
\begin{exercise}
	$\iint_D e^{x/y}dxdy$,其中$D$是由抛物线$y^2=x$,直线$x=0,y=1$所围成;
\end{exercise}
\begin{exercise}
	$\iint_D ydxdy$,其中$D$是圆$x^2 + y^2 \le ax$与$x^2 +y^2 \le ay$的公共部分$\left( a>0 \right)$
\end{exercise}
\begin{exercise}
	$\iint_D \left( x+y+2y^2 \right)dxdy$,其中$D$是由圆周$x^2+y^2=2ax$所围成的区域$\left( a>0 \right)$
\end{exercise}
\begin{exercise}
	改变下列积分的积分顺序:\\
	\begin{enumerate}
		\item $\int_0^adx\int_x^{\sqrt{2ax-x^2}}f \left( x,y \right)dy$
		\item $\int_{-6}^2dy\int_{\frac{y^2}{4}-1}^{2-y}f \left( x,y \right)dx$
	\end{enumerate}
\end{exercise}
\end{document}


