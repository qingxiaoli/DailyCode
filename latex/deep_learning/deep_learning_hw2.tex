%!TEX program = xelatex
\documentclass[a4paper, UTF8]{ctexrep}
\usepackage{ctex}
\usepackage{amsmath}
\usepackage{multirow}
\usepackage{amssymb}
\usepackage{graphicx}
\usepackage{geometry}
\usepackage{bm}
\usepackage{subfigure}
\usepackage{float}
\usepackage{array}
\usepackage{makecell}

\renewcommand\thesection{\arabic{section}}

\begin{document}
	\begin{titlepage}
		\centering
		\vspace{6cm}
		\LARGE{\textbf{Deep Learning Homework 2}}\\
		\vspace{4cm}
		\includegraphics[width=0.8\textwidth]{deepLearning.png}\\
		\vspace{4cm}
		\normalsize{安捷 1601210097}\\
		\normalsize{\today}
	\end{titlepage}
		\section{算法实现简介}
			在这次作业中,为了能够更好的实验不同优化算法的性能及对参数的敏感性,我放弃了上次作业中使用的MLP网络,重新实现了一个有两个卷积层和一个全连接层的CNN,并针对mnist数据集测试了不同优化算法的性能,性能测试主要分为两部分:
			\begin{enumerate}
				\item 使用每个算法默认的学习率及其他超参数,分别测试算法达到99\%测试准确率的迭代次数,以此来显示算法的性能;
				\item 对每一个算法,分别使用默认学习率或超参数、比默认参数高一个数量级、低一个数量级三组参数分别测试其在上一步达到99\%准确率的迭代次数时达到了准确率,以此作为衡量算法对学习率或超参数敏感性的指标;
			\end{enumerate}
		\section{算法性能测试}
			\begin{table}[htbp!]
				\centering
				\begin{tabular}{cccc}
					\hline
					算法名称 & 学习率/超参数 & 迭代次数 & 备注 \\
					\hline
					SGD & 0.001 & 4000 &   \\

				\end{tabular}
				\caption{优化算法达到99\%准确率所需迭代次数}
			\end{table}
		\section{算法超参数稳定性测试}
			\begin{table}[htbp!]
				\centering
				\begin{tabular}{cccc}

				\end{tabular}
				\caption{算法不同超参数下固定迭代次数达到的准确率}
			\end{table}


    \section{代码运行环境及测试平台信息}
      \begin{table}[htbp!]
        \centering
        \begin{tabular}{l}
          \hline
          Python Version: 3.6.0 \\
          Tensorflow Version: tensorflow-gpu-1.0.1 \\
          CUDA Version: 8.0 \\
          OS: Arch Linux \\
          Kernel: x86\_64 Linux 4.10.4-1-ARCH \\
          CPU: Intel Core i7-6700K @ 8x 4.2GHz \\
          GPU: GeForce GTX 1060 6GB \\
          RAM: 16003MiB \\
          \hline
        \end{tabular}
        \caption{代码运行环境及测试环境表}
      \end{table}
      在没有NVIDIA\ GPU及CUDA支持的环境下代码依然可以运行,只是速度较慢
    \section{总结}
      通过这次作业,我学习了tensorflow的基本使用方法,在实现了MLP之后,我又实现了CNN并成功部署到了科研实验中,接下来我将尝试用tensorflow实现FCN用于图像分割;在完成作业的过程中,我同时利用上述测试环境中的CPU与GPU进行了训练,发现GPU训练的速度大约是CPU训练速度的两倍,之后的作业在显存足够的情况下我将主要使用GPU进行训练。
\end{document}
