%!TEX program = xelatex
\documentclass[a4paper, UTF8]{ctexrep}
\usepackage{ctex}
\usepackage{amsmath}
\usepackage{multirow}
\usepackage{amssymb}
\usepackage{graphicx}
\usepackage{geometry}
\usepackage{bm}
\usepackage{subfigure}
\usepackage{float}
\usepackage{array}
\usepackage{makecell}

\renewcommand\thesection{\arabic{section}}

\begin{document}
    \begin{titlepage}
        \centering
        \vspace{6cm}
        \LARGE{\textbf{Deep Learning Homework 4}}\\
        \vspace{4cm}
        \includegraphics[width=0.8\textwidth]{deepLearning.png}\\
        \vspace{4cm}
        \normalsize{安捷 1601210097}\\
        \normalsize{\today}
    \end{titlepage}
        \section{算法实现简介}
        	在这次作业中,我使用tensorflow github中的cifar10 cnn代码,进行了一定的修改和调参,希望可以使得算法达到最好的性能。
        \section{数值实验结果}
          我做了四组数值实验,下面分别将结果展示如下:
          \subsection{2conv-32-64,ite=50000}
            2017-04-21 12:39:10.283589: step 49950, loss = 0.76 (1050.7 examples/sec; 0.122 sec/batch)
            2017-04-21 12:39:11.515913: step 49960, loss = 0.75 (1038.7 examples/sec; 0.123 sec/batch)
            2017-04-21 12:39:12.732593: step 49970, loss = 0.58 (1052.0 examples/sec; 0.122 sec/batch)
            2017-04-21 12:39:13.975698: step 49980, loss = 0.59 (1029.7 examples/sec; 0.124 sec/batch)
            2017-04-21 12:39:15.223052: step 49990, loss = 0.68 (1026.2 examples/sec; 0.125 sec/batch)

            2017-04-21 12:57:15.110323: precision @ 1 = 0.854

          \subsection{2conv-32-64,ite=438860}
            2017-04-21 10:49:53.340448: step 438820, loss = 0.12 (1148.0 examples/sec; 0.111 sec/batch)
            2017-04-21 10:49:54.463862: step 438830, loss = 0.11 (1139.4 examples/sec; 0.112 sec/batch)
            2017-04-21 10:49:55.602834: step 438840, loss = 0.11 (1123.8 examples/sec; 0.114 sec/batch)
            2017-04-21 10:49:56.711436: step 438850, loss = 0.11 (1154.6 examples/sec; 0.111 sec/batch)
            2017-04-21 10:49:57.837942: step 438860, loss = 0.11 (1136.3 examples/sec; 0.113 sec/batch)

            2017-04-21 10:50:14.644544: precision @ 1 = 0.870 

          \subsection{2conv-32-32-64,ite=50000}
            2017-04-21 20:23:15.227485: step 49950, loss = 0.77 (1342.8 examples/sec; 0.095 sec/batch)
            2017-04-21 20:23:16.157790: step 49960, loss = 0.50 (1375.9 examples/sec; 0.093 sec/batch)
            2017-04-21 20:23:17.112762: step 49970, loss = 0.63 (1340.3 examples/sec; 0.095 sec/batch)
            2017-04-21 20:23:18.056657: step 49980, loss = 0.62 (1356.1 examples/sec; 0.094 sec/batch)
            2017-04-21 20:23:19.001197: step 49990, loss = 0.66 (1355.1 examples/sec; 0.094 sec/batch)

            2017-04-21 21:06:45.512474: precision @ 1 = 0.834

          \subsection{2conv-32-32-64,ite=200000}
            2017-04-22 02:35:04.625261: step 199950, loss = 0.32 (1304.0 examples/sec; 0.098 sec/batch)
            2017-04-22 02:35:05.628415: step 199960, loss = 0.40 (1276.0 examples/sec; 0.100 sec/batch)
            2017-04-22 02:35:06.630871: step 199970, loss = 0.27 (1276.9 examples/sec; 0.100 sec/batch)
            2017-04-22 02:35:07.636670: step 199980, loss = 0.30 (1272.6 examples/sec; 0.101 sec/batch)
            2017-04-22 02:35:08.615497: step 199990, loss = 0.27 (1307.7 examples/sec; 0.098 sec/batch)

            2017-04-22 11:42:12.524210: precision @ 1 = 0.853
    	\section{代码运行环境及测试平台信息}
      \begin{table}[htbp!]
        \centering
        \begin{tabular}{l}
          \hline
          Python Version: 3.6.0 \\
          Tensorflow Version: tensorflow-gpu-1.0.1 \\
          CUDA Version: 8.0 \\
          OS: Arch Linux \\
          Kernel: x86\_64 Linux 4.10.4-1-ARCH \\
          CPU: Intel Core i7-6700K @ 8x 4.2GHz \\
          GPU: GeForce GTX 1060 6GB \\
          RAM: 16003MiB \\
          \hline
        \end{tabular}
        \caption{代码运行环境及测试环境表}
      \end{table}
      在没有NVIDIA\ GPU及CUDA支持的环境下代码依然可以运行,只是速度较慢
    \section{总结}
      针对cifar10数据集,cnn的合适规模大约是两个卷积层,两层卷积层在500000次左右迭代的效果最好,过大的规模会降低测试集准确度。
\end{document}