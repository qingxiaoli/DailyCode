\documentclass[12pt, oneside]{article}
\usepackage{geometry}
\geometry{a4paper}
\geometry{left=2cm, right=2cm,top=2cm, bottom=2cm}
\usepackage{graphicx}
\usepackage{amssymb}
\usepackage{CJKutf8}
\usepackage{amsmath}
\usepackage{amsthm} 
\usepackage{bm}
%\usepackage{CTEX}

\newtheorem*{exercise}{\textbf{习题}}

\begin{CJK}{UTF8}{gbsn}
  \title{Manifold Learning Homework 1}
  \author{安捷 1601210097}
  \date{2017-02-25}
  \begin{document}
  \maketitle

  \begin{exercise}[1]
	\begin{equation*} 
	  \operatorname{tr}\left(\rm{A} \right) = 15
	\end{equation*}
  \end{exercise}

  \begin{exercise}[2]
	\scriptsize
	\begin{equation*}
	  \rm{A} = 
	  \left( 
	  \begin{array}{ccc}
		0.9487 & -0.1778 & 0.2615 \\
		0 & 0.8269 & 0.5623 \\
		-0.3126 & -0.5335 & 0.7845 \\
	  \end{array}
	  \right)
	  \left( 
	  \begin{array}{ccc}
		0 & 0 & 0 \\
		0 & 0.6993 & 0 \\
		0 & 0 & 14.3007 \\
	  \end{array}
	  \right)
	  \left( 
	  \begin{array}{ccc}
		0.9487 & 0 & -0.3126 \\
		-0.1778 & 0.8269 & -0.5335 \\
		0.2615 & 0.5623 & 0.7845 \\
	  \end{array}
	  \right)
	\end{equation*}
  \end{exercise}

  \begin{exercise}[3]
	\begin{equation*}
	  \scriptsize
	  \rm{
		A_{Full-SVD} = 
		\left( 
		\begin{array}{cccc}
		  -0.1409 & 0.8247 & 0.5477 & -0.0037 \\
		  -0.3439 & 0.4263 & -0.1880 & 0.4131 \\
		  -0.5470 & 0.0278 & -0.1880 & -0.8153 \\
		  -0.7501 & -0.3706 & 0.3679 & 0.4058 \\
		\end{array}
		\right)
		\left( 
		\begin{array}{ccc}
		  25.4624 & 0 & 0 \\
		  0 & 1.2907 & 0 \\
		  0 & 0 & 0 \\
		  0 & 0 & 0 \\
		\end{array}
		\right)
		\left( 
		\begin{array}{ccc}
		  -0.5045 & -0.7608 & -0.4092 \\
		  -0.5745 & -0.0571 & 0.8165 \\
		  -0.6445 & 0.6465 & -0.4082 \\
		\end{array}
		\right)
	  }
	\end{equation*} \\	
	\begin{equation*}
	  \scriptsize
	  \rm{
		A_{Thin-SVD} = 
		\left( 
		\begin{array}{ccc}
		  -0.1409 & 0.8247 & 0.5477 \\
		  -0.3439 & 0.4263 & -0.7276 \\
		  -0.5470 & 0.0278 & -0.1880 \\
		  -0.7501 & -0.3706 & 0.3679 \\
		\end{array}
		\right)
		\left( 
		\begin{array}{ccc}
		  25.4624 & 0 & 0 \\
		  0 & 1.2907 & 0 \\
		  0 & 0 & 0 \\
		\end{array}
		\right)
		\left( 
		\begin{array}{ccc}
		  -0.5045 & -0.7608 & -0.4092 \\
		  -0.5745 & -0.0571 & 0.8165 \\
		  -0.6445 & 0.6465 & -0.4082 \\
		\end{array}
		\right)
	  }
	\end{equation*}
  \end{exercise}

  \begin{exercise}[4]
	\scriptsize
	\begin{equation*}
	  \rm{
		P_x = \left(
		\begin{array}{cccc}
		  0.4794 & -0.1850 & -0.2398 & 0.3973 \\
		  -0.1850 & 0.9343 & -0.0852 & 0.1412 \\
		  -0.2398 & -0.0852 & 0.8896 & 0.1830 \\
		  0.3973 & 0.1412 & 0.1830 & 0.6968 \\
		\end{array}
		\right)
	  }
	\end{equation*}
  \end{exercise}

  \begin{exercise}[5]
	\rm{
	  \textbf{Proof}:\\
	  首先,考虑矩阵$\rm{AXB}$的第$k$列:
	  \begin{equation*}
		\left( \rm{AXB} \right)_k = \rm{AXB}_k
	  \end{equation*}
	  \begin{equation*}
		= \rm{A} \left( x_1, x_2, x_3, \dots , x_p \right)
		\left(
		\begin{array}{c}
		  b_{1k} \\
		  b_{2k} \\
		  b_{3k} \\
		  . \\
		  . \\
		  . \\
		  b_{pk} \\
		\end{array}
		\right)
	  \end{equation*}
	  \begin{equation*}
		= \left( b_{1k}\rm{A}, b_{2k}\rm{A}, b_{3k}\rm{A}, \dots , b_{pk}\rm{A}\right) \left( 
		\begin{array}{c}
		  x_1 \\
		  x_2 \\
		  x_3 \\
		  . \\
		  . \\
		  . \\
		  x_p \\
		\end{array}
		\right)
		\end{equation*}
		\begin{equation*}
		  = \left( b_{1k}, b_{2k}, b_{3k}, \dots , b_{pk}\right)\rm{A}\operatorname{Vec}\left( \rm{X} \right)
		\end{equation*}
		\begin{equation*}
		  = \left( \rm{B}_k^T \otimes \rm{A} \right) \operatorname{Vec} \left( \rm{X} \right)
		  \end{equation*}
		  而这一结果正好是最终$\rm{Vec}$之后向量对应于开始选取k的行数,需要注意的是,这里的$k$并不直接对应与最后$\rm{Vec}$中的行数,而是$\rm{AXB}$经过$\rm{Vec}$之后对应的某一个k行,当然,这样选取k最终可以选到$\rm{Vec} \left( \rm{AXB} \right)$的所有行
		  \begin{equation*}
			\therefore \operatorname{Vec}\left( \rm{AXB} \right)	= \left( \rm{B}^T \otimes \rm{A} \right) \operatorname{Vec} \left( \rm{X} \right)
		  \end{equation*}
		}
	  \end{exercise}

	  \begin{exercise}[8]
		\rm{
		  \textbf{Proof}: \\
		  两边同乘$\left( \rm{A} + UCV^T \right)^{-1}$,有:\\
		  \begin{equation*}
			\begin{aligned}
			  \rm{I} &= \rm{I - U \left( C^{-1} + V^T \rm{A}^{-1} U \right)^{-1} V^T \rm{A}^{-1}} \\
			  &+ \rm{UCV^T \rm{A}^{-1} - UCV^T \rm{A}^{-1} U \left( C^{-1} + V^T \rm{A}^{-1} U\right) V^T \rm{A}^{-1}}
			\end{aligned}
		  \end{equation*}
		  故只需有\\
		  \begin{equation*}
			\rm{
			  -U \left( C^{-1} + V^T \rm{A}^{-1} U \right)^{-1} + UC - UCV^T \rm{A}^{-1} U \left( C^{-1} + V^T \rm{A}^{-1} U\right)^{-1} = 0
			}
		  \end{equation*}
		  即\\
		  \begin{equation*}
			\rm{
			  C - \left( C{-1} + V^T \rm{A}^{-1} U \right)^{-1} - CV^TA^{-1}U \left( C^{-1} + V^T \rm{A}^{-1}U \right)^{-1} = 0
			}
		  \end{equation*}
		  \begin{equation*}
			\rm{
			  C - \left(I + CV^TA^{-1}U \right) \left( C{-1} + V^T \rm{A}^{-1} U \right)^{-1} = 0
			}
		  \end{equation*}
		  而\\
		  \begin{equation*}
			\rm{
			  I + CV^T \rm{A}^{-1} U = C \left( C^{-1} + V^T \rm{A}^{-1} U \right)
			}
		  \end{equation*}
		  故原式得证
		}
	  \end{exercise}

	  \begin{exercise}[21]
		\begin{description}
		  \item[1-norm] = $30$ 
		  \item[2-norm] = $25.4624$
		  \item[$\infty$-norm] = $33$
		  \item[F-norm] = $25.4951$
		  \item[Nuclear-norm] = $26.7531$
		  \item[(2,1)-norm] = $43.0445$
		\end{description}	
	  \end{exercise}

	\end{CJK}
	\end{document}  
