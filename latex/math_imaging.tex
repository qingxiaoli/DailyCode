\documentclass[a4paper, UTF8]{ctexart}
\usepackage{ctex}
\usepackage{amsmath}
\usepackage{amsthm}
\usepackage{multirow}
\usepackage{amssymb}
\usepackage{graphicx}
\usepackage{geometry}
\usepackage{bm}
\usepackage{subfigure}
\usepackage{float}
\usepackage{mathrsfs}

\renewcommand\thesection{\arabic{section}}
\newtheorem*{exercise}{\textbf{习题}}
\newtheorem*{theorem}{Theorem}

\title{Manifold Learning Homework 3}
\date{2017-03-11}
\author{安捷 1601210097}

\begin{document}
\maketitle
  \begin{exercise}
    对于图像中位置为 $\left( i, j \right)$ 的点来说,不妨认为这一点就位于 $0,1$的边界上,切这一点的数值为 $0$,实际上对于任一点以下推导都是成立的,且边界上的点一定可以在本题中求出 $\lambda$的值。对于第一种梯度的定义方式,在 $0$度边界的情形有
    \begin{equation*}
      \frac{\partial v}{\partial x}\Big |_{i,j} = \frac{1}{2}
    \end{equation*}
    \begin{equation*}
      \frac{\partial v}{\partial y}\Big |_{i,j} = 0
    \end{equation*}
    所以, $\lVert \nabla v \rVert = \frac{1}{2}$。\\
    在 $45$度边界的情形有
    \begin{equation*}
      \frac{\partial v}{\partial x}\Big |_{i,j} = \frac{1}{2}
    \end{equation*}
    \begin{equation*}
      \frac{\partial v}{\partial y}\Big |_{i,j} = \frac{1}{2}
    \end{equation*}
    所以, $\lVert \nabla v \rVert = \frac{1}{\sqrt{2}}$。
    对于第二种离散梯度的定义方式,要使二者相同。首先,在 $0$度的情形,有
    \begin{equation*}
      \frac{\partial v}{\partial x}\Big |_{i,j} = \frac{\lambda}{2} + \frac{1-\lambda}{2} = \frac{1}{2}
    \end{equation*}
    \begin{equation*}
      \frac{\partial v}{\partial y}\Big |_{i,j} = 0
    \end{equation*}
    所以, $\lVert \nabla v \rVert = \frac{1}{2}$。
    在 $45$度边界的情形有
    \begin{equation*}
      \frac{\partial v}{\partial x}\Big |_{i,j} = \frac{\lambda}{2} + \frac{1-\lambda}{4} = \frac{1+\lambda}{4}
    \end{equation*}
    \begin{equation*}
      \frac{\partial v}{\partial y}\Big |_{i,j} = -\frac{\lambda}{2} - \frac{1-\lambda}{4} = -\frac{1+\lambda}{4}
    \end{equation*}
    所以, $\lVert \nabla v \rVert = \frac{\sqrt{2}\left(1+\lambda\right)}{4}$。当 $\lambda = \sqrt{2} - 1$时, $45$度与 $0$度有相同的范数。
  \end{exercise}
  \begin{exercise}
    与上题的分析方法相同,在 $0$度的情形,有
    \begin{equation*}
      \Delta v\big |_{i,j} = \lambda + \left( 1 - \lambda \right) = 1
    \end{equation*}
    在$45$ 度的情形,有
    \begin{equation*}
      \Delta v\big |_{i,j} = 2\lambda + \frac{1-\lambda}{2} = \frac{1 + 3 \lambda}{2}
    \end{equation*}
    当 $\lambda = \frac{1}{3}$时, $45$ 度与 $0$ 度有相同的值。
  \end{exercise}
\end{document}
